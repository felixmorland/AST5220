\documentclass{aa}  

\usepackage{graphicx}
\usepackage{bookmark}
\usepackage{txfonts}
\usepackage{todonotes}
% \usepackage[options]{hyperref}

\renewcommand{\b}[1]{\mathbf{#1}}
\renewcommand{\d}{\text{d}}
\newcommand{\bivec}[1]{\overset{\leftrightarrow}{#1}}
\newcommand{\dd}[2]{\frac{\d #1}{\d #2}}
\newcommand{\pd}[2]{\frac{\partial #1}{\partial #2}}
\newcommand{\ub}[2]{\underbrace{#1}_{#2}}

\begin{document} 

   \title{Calculating the CMB power spectrum from simulated cosmic structure formation initiated by primordial fluctuations}

   \author{F. Morland\inst{1}}

   \institute{Institute of Theoretical Astrophysics,  
                University of Oslo,  0315 Oslo,  Norway\\
              \email{felix.morland@astro.uio.no}
             }

   \date{}

  \abstract{An abstract for the paper. Describe the paper. What is the paper about, what are the main results, etc.}

   \keywords{   cosmic microwave background  --
                large-scale structure of Universe
               }

   \maketitle

\section{Introduction}
Write an introduction here. Give context to the paper. Citations to relevant papers. You only need to do this in the end for the last milestone.  

\section{Milestone I: Background cosmology}
To calculate the CMB power spectrum, we evolve primordial perturbations from the early Universe to the present day. Thus we must begin by constructing the background cosmology, into which we then can introduce perturbations. Our model Universe will have the possibility to contain two relativistic components (photons and neutrinos), two non-relativistic matter components (baryonic and CDM), a general (possibly non-zero) curvature density parameter $-1\le \Omega_{k0} \le 1$, and a cosmological constant. 

We first develop the theoretical framework describing the temporal evolution of density components in the Universe, along with derived quantities for model validation. We then employ an MCMC algorithm to constrain cosmological parameters using luminosity distance measurements from supernova observations.

\todo[inline, color=red]{
      Equality times\\
      Acceleration time\\
      Age of the Universe; spline $t(x)$\\
      Plot $d_L(z)/z$ vs. supernova data\\ 
      Compare $\mathcal{H}'/\mathcal{H}$ and $\mathcal{H}''/\mathcal{H}$ to expectations\\
      Posterior for $h$
}

\subsection{Theory}
\todo[inline, color=red]{
      Einstein field eqs.\\
      FLRW metric and the Friedmann eqs.\\
      Hubble function\\
      Evolution of density parameters\\
      Cosmic time\\
      Conformal time\\
      Luminosity distance
}

\begin{align*}
      \dd{\mathcal{H}}{x} = \mathcal{H}\bigg[1- \frac{3}{2}\Omega_{m} -2\Omega_{r} - \Omega_k\bigg] \;\Bigg|_{a = e^x}
\end{align*}

\begin{align*}
      \dd{^2\mathcal{H}}{x^2} =&\; \mathcal{H}\Bigg\{\bigg[1-\frac{3}{2}\Omega_m -2\Omega_r  -\Omega_k\bigg]^2\\
      & -\frac{1}{2}\Big(3\Omega_m + 4\Omega_r + 2\Omega_k \Big)^2+\frac{1}{2}\Big(9\Omega_m +16\Omega_r +4\Omega_k\Big)\Bigg\} \;\Bigg|_{a = e^x}
\end{align*}

\begin{align*}
      x^{mr}_\text{eq} = \ln\bigg(\frac{\Omega_{r0}}{\Omega_{m0}}\bigg),\quad\quad x^{\Lambda m}_\text{eq} = \frac{1}{3}\ln\bigg(\frac{\Omega_{m0}}{\Omega_{\Lambda 0}}\bigg)
\end{align*}



\subsection{Implementation details}
\subsubsection{MCMC parameter fitting}
\begin{equation*}
      \chi^2(h,\Omega_{m0}, \Omega_{k0}) = \sum_{i=1}^N \frac{[d_L(z_i,h,\Omega_{m0},\Omega_{k0}) - d_L^{\rm obs}(z_i)]^2 }{\sigma_i^2}
\end{equation*}

Something about the numerical work 
Minimum chi2 found: chi 29.2799, h 0.701711, OmegaM0 0.255027, OmegaK0 0.0789514
t0 = 13.6781 Gyr




\subsection{Results}
Show and discuss the results. 

\begin{figure}
      \centering
      \includegraphics[width=0.9\linewidth]{../figures/dL_obs_vs_pred.pdf}
      \caption{
            Luminosity distance as a function of redshift $z$. The data points represent the luminosity distances deduced from observations of supernovae at different redshifts. The two curves show the luminosity distance functions $d_L(z)/z$ for the best fit parameters and for the fiducial cosmological parameters.
      }
      \label{fig:0}
\end{figure}

\begin{figure}
      \centering
      \includegraphics[width=0.9\linewidth]{../figures/supernova_fitting.pdf}
      \caption{
            The regions of MCMC sampled points in the $(\Omega_{M0}, \Omega_{\Lambda0})$ parameter space within $1\sigma$ and $2\sigma$ of the pair associated with the minimal $\chi^2$. The diagonal dashed line shows the cosmologies giving a flat Universe; $\Omega_{K0}= 1-\Omega_{M0}-\Omega_{\Lambda0} = 0$.
      }
      \label{fig:1}
\end{figure}
\begin{figure}
      \centering
      \includegraphics[width=0.9\linewidth]{../figures/bestfit_OmegaLambda.pdf}
      \caption{
            k
      }
      \label{fig:2}
\end{figure}
\begin{figure}
      \centering
      \includegraphics[width=0.9\linewidth]{../figures/Hp_derivatives.pdf}
      \caption{
            yes
      }
      \label{fig:3}
\end{figure}
\begin{figure*}
      \centering
      \includegraphics[width=0.9\linewidth]{../figures/test_background_cosmology.pdf}
      \caption{
            uhm
      }
      \label{fig:4}
\end{figure*}

% \section{Milestone II}
% Some introduction about what it is all about.

% \subsection{Theory}
% The theory behind this milestone.

% \subsection{Implementation details}
% Something about the numerical work.

% \subsection{Results}
% Show and discuss the results.

% \section{Milestone III}
% Some introduction about what it is all about.

% \subsection{Theory}
% The theory behind this milestone.

% \subsection{Implementation details}
% Something about the numerical work.

% \subsection{Results}
% Show and discuss the results.

% \section{Milestone IV}
% Some introduction about what it is all about.

% \subsection{Theory}
% The theory behind this milestone.

% \subsection{Implementation details}
% Something about the numerical work.

% \subsection{Results}
% Show and discuss the results.

% \section{Conclusions}

% Write a short summary and conclusion in the end. 

\begin{acknowledgements}
      I thank my mom for financial support!
\end{acknowledgements}

\begin{thebibliography}{}

  \bibitem[Baker(1966)]{baker} Baker, N. 1966,
      in Stellar Evolution,
      ed.\ R. F. Stein,\& A. G. W. Cameron
      (Plenum, New York) 333
\end{thebibliography}

\begin{appendix}
\section{Derivatives of the conformal Hubble function}

The Hubble function is given by
\begin{equation*}
H = H_0 \sqrt{(\Omega_{b0}+\Omega_{\rm CDM 0})a^{-3} + (\Omega_{\gamma 0} + \Omega_{\nu 0}) a^{-4} + \Omega_{k 0} a^{-2} + \Omega_{\Lambda 0}},
\end{equation*}
where we can define $\Omega_{m0}\equiv \Omega_{b0}+\Omega_{\rm CDM 0}$ and $\Omega_{r0}\equiv \Omega_{\gamma 0} + \Omega_{\nu 0}$. The first derivative of the conformal Hubble function $\mathcal{H}\equiv aH(a)$ with respect to $x\equiv\ln a$ is
\begin{align*}
      \dd{\mathcal{H}}{x} &= \dd{a}{x}H + a \dd{H}{x} = \mathcal{H} + a^2 \dd{H}{a}.
\end{align*}
Let us compute the derivative of $H(a)$:
\begin{align*}
      \dd{H}{a} = -\frac{H_0^2}{2H}\Big(3\Omega_{m0}a^{-4} + 4\Omega_{r 0} a^{-5} + 2\Omega_{k 0} a^{-3}\Big).
\end{align*}
This gives 
\begin{align*}
      \dd{\mathcal{H}}{x} &= \mathcal{H} - a^2 \frac{H_0^2}{2H}\Big(3\Omega_{m0}a^{-4} + 4\Omega_{r 0} a^{-5} + 2\Omega_{k 0} a^{-3} \Big)\\
      &= \mathcal{H}\bigg[1- \frac{3}{2}\Omega_{m} -2\Omega_{r} - \Omega_k\bigg].
\end{align*}
Here the $\Omega_i$'s refer to the density parameters as functions of scale factor, i.e. $\Omega_i(a)$, which should be evaluated at $a=e^x.$\\

The second derivative is a bit more cumbersome to calculate; from the product rule it follows that 
\begin{align*}
      \dd{^2\mathcal{H}}{x^2} =& \;\mathcal{H}\bigg[1- \frac{3}{2}\Omega_{m} -2\Omega_{r} - \Omega_k\bigg]^2\\
      &+ \mathcal{H}\dd{}{x}\bigg[1- \frac{3}{2}\Omega_{m} -2\Omega_{r} - \Omega_k\bigg].
\end{align*}
Let $B$ be the expression inside the big bracket. We thus need $\d B/\d x$, which is most easily calculated by factoring out $H_0^2/H(a)^2$ from the last three terms of $B$;
\begin{align*}
      \dd{B}{x} =& -\frac{a}{2}\dd{}{a}\Bigg(\frac{H_0^2}{H^2}\Bigg)\Big(3\Omega_{m0}a^{-3} + 4\Omega_{r0}a^{-4} + 2\Omega_{k0}a^{-2} \Big)\\
      &-\frac{a}{2}\frac{H_0^2}{H^2}\dd{}{a}\Big(3\Omega_{m0}a^{-3} + 4\Omega_{r0}a^{-4} + 2\Omega_{k0}a^{-2} \Big)\\[3mm]
      =& -\frac{1}{2}\Big(3\Omega_m +4\Omega_r +2\Omega_k\Big)^2 + \frac{1}{2}\Big(9\Omega_m + 16 \Omega_r + 4 \Omega_k\Big).
\end{align*}
From this we end up with second derivative
\begin{align*}
      \dd{^2\mathcal{H}}{x^2} =&\; \mathcal{H}\Bigg\{\bigg[1-\frac{3}{2}\Omega_m -2\Omega_r  -\Omega_k\bigg]^2\\
      & -\frac{1}{2}\Big(3\Omega_m + 4\Omega_r + 2\Omega_k \Big)^2+\frac{1}{2}\Big(9\Omega_m +16\Omega_r +4\Omega_k\Big)\Bigg\} \;\Bigg|_{a = e^x}.
\end{align*}

\end{appendix}

\end{document}